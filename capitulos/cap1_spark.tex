\chapter{¿Qué es Spark?}
Responder a esta pregunta sin divagaciones es posible: Apache Spark es un motor de código abierto de computación en memoria unificado, con un conjunto de bibliotecas para el procesamiento paralelo de datos en clústeres de computadoras. Sin embargo, no sería lícito para su entramado significativo que no nos adentrásemos en tal definición, pues Spark  posee varias características  relevantes para el mundo de la computación.\\

Veamos algunas de sus propiedades:\\

\begin{itemize}
\item Es compatible con Python, Java, R y Scala, su lenguaje nativo.\\

\item Utiliza commodity hardware, lo que supone que pueda ser utilizado desde en un ordenador portátil a en clúster de miles de servidores, pues no requiere de computadores avanzados para su utilización.\\

\item Tiene la capacidad de realizar la mayoría de las operaciones en memoria (sin necesidad de escribir a disco). Por esta razón, podemos afirmar que está dotado de mayor rapidez de la que tienen otras opciones, como HadoopMapReduce.\\

\item Ofrece una plataforma unificada que admite una amplia gama de tareas para el análisis de datos, como la carga de datos en crudo, las consultas SQL, el machine learning, el cómputo en streaming o el tratamiento de grafos, todo esto a través del mismo motor de computación y con un conjunto consistente de APIs. \\

\item La filosofía que persigue Spark con esta visión unificada es la de la combinación de diferentes tipos de bibliotecas y procesamientos dispares, tal y como viene exigiendo el mundo real. Referente a las bibliotecas, cabe decir que además de las bibliotecas estándares, de las que hablamos en el siguiente punto, Spark también admite bibliotecas externas publicadas por las comunidades de código abierto.\\

\item Las bibliotecas estándares de Spark, son la mayor parte del proyecto de código abierto. Pese a que las bibliotecas han crecido para ampliar su funcionalidad, el motor central Spark (Sparkcore) apenas se ha modificado desde la fecha de su lanzamiento.\\

\item Su estructura homogénea permite realizar análisis de manera más simple y más eficiente.\\

\item Es importante señalar que Spark, como motor de computación, no tiene como fin almacenar datos, sino manejar la carga de datos de sistemas de almacenamiento y la realización de cálculos.\\

\item Puede ser utilizado en una gran variedad de sistemas de almacenamiento persistentes, incluidos sistemas de archivos distribuidos, como Hadoop HDFS, buses de mensajes, como Apache Kafka, o sistemas de almacenamiento en la nube, como Azure Storage o Amazon S3.\\

\end{itemize}
